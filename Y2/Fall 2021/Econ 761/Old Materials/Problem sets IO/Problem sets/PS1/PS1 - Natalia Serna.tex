\documentclass[11pt,english]{article}
\usepackage[T1]{fontenc}
\usepackage{babel}
\usepackage[margin=0.8 in]{geometry}
\usepackage{caption}
\usepackage{subfig}
\usepackage{longtable}
\usepackage{natbib}
\linespread{1.15}
\usepackage{tikz}
\usepackage{setspace}
\usepackage{multirow}
\usepackage{multicol}
\usepackage{csquotes}
% ------
% Fonts and typesetting settings


%\usepackage[sc]{mathpazo}
\usepackage{titling}									
\usepackage{float}
\usepackage{pdflscape}
\usepackage[toc]{appendix}
\renewcommand{\appendixtocname}{Appendices}
%\renewcommand{\appendixsection}{\normalfont\bfseries}
\usepackage{amsmath, amsthm, amsfonts}


\newcommand{\subtitle}[1]{%
  \posttitle{%
    \par\end{center}
    \begin{center}\large#1\end{center}
    \vskip0.5em}%
}

\usepackage{booktabs}												
\usepackage{natbib}                                                 
\usepackage{graphics,epsfig}						
% -----
% ------



% Maketitle metadata
\author{
Natalia Serna
   }
 \title{Problem set 1}

\date{}
%%%%%%%%%%%%%%%%%%%%%%%%
\begin{document}

\maketitle

\section{Problem 1}

\begin{enumerate}
\item Risk neutral bidders with common value and first-price sealed bid auction. In this setting the normal form game is:

\begin{itemize}
\item Set of players $\mathcal{P}=\{1,2\}$
\item Set of strategies $b_{i}:v\to [0,v]$, $\forall i\in\{1,2\}$ 
\item Payoffs:
\begin{equation}
\nonumber
U_{i}(b_{i}, b_{j})=\left\{
\begin{array}{ccc}
v_{i}-b_{i}& \mbox{if}& b_{i}<b_{j}\\
(v_{i}-b_{i})/2& \mbox{if}& b_{i}=b_{j}\\
0 & \mbox{if}& b_{i}>b_{j}\\
\end{array}\right.
\end{equation}
\end{itemize}


\textcolor{blue}{\textbf{Claim}: There exists a unique symmetric equilibrium with $b_{i}=b_{j}=v$}

\textbf{Proof}:
\begin{itemize}
\item By way of contradiction suppose $b_{i}=b_{j}=v/2<v$. Then $E[u_{i}|b_{i}, b_{j}]=v/4<E[u_{i}|v/2+\epsilon, b_{j}]=v/2-\epsilon$ for sufficiently small $\epsilon$. Therefore, each bidder has incentives to deviate.
\item We know none of the bidders want to bid above their valuation because their expected payoff would be negative.
\item If $b_{i}=b_{j}=v$, then $E[u_{i}|b_{i}, b_{j}]=0$ and no bidder has incentives to deviate.
\end{itemize}

\item For the all-pay auction the payoffs are
\begin{equation}
\nonumber
U_{i}(b_{i}, b_{j})=\left\{
\begin{array}{ccc}
v_{i}-b_{i}& \mbox{if}& b_{i}<b_{j}\\
(v_{i}-b_{i})/2& \mbox{if}& b_{i}=b_{j}\\
-b_{i} & \mbox{if}& b_{i}>b_{j}\\
\end{array}\right.
\end{equation}

\item \textcolor{blue}{Suppose $v_{i}=v_{j}=v$, then there does not exist a pure strategy Nash equilibrium.}

\textbf{Proof}:
\begin{itemize}
\item Pick a pair of bids $(b_{1}, b_{2})$ such that $b_{2}>v_{2}=v_{1}>b_{1}$. We can immediately rule out this as an equilibrium since it is not rational to bid above one's valuation. By symmetry we can also rule out regions where $b_{1}>v_{1}=v_{2}>b_{2}$.
\item Pick a pair of bids $(b_{1}, b_{2})$ such that $v_{1}=v_{2}>b_{2}>b_{1}$. Then, player 2 has incentives to deviate and bid $b_{2}'=b_{1}+\epsilon$, since the probability of winning is still one and the expected payoff is higher than bidding $b_{2}$. So there is no Nash equilibrium in this region. By symmetry the same argument holds for the region where $v_{1}=v_{2}>b_{1}>b_{2}$.
\item Pick a pair of bids $(b_{1}, b_{2})$ such that $v_{1}=v_{2}=b_{2}=b_{1}$. Then player $i$ has incentives to bid $b_{i}'=b_{i}+\epsilon$ in which case the probability of winning is one and the expected payoff $E[u_{i}|b_{i}']=v_{i}-b_{i}-\epsilon>E[u_{i}|b_{i}]=(v_{i}-b_{i})/2$, for sufficiently small $\epsilon$.
\end{itemize}

\item In the mixed strategy equilibrium we want to find the distribution that each player uses to make the other indifferent. By lemma 2 such distributions are defined on an interval ${[\underline{b},\overline{b}]}$ in which they are right continuous and have no simultaneous mass points in the $b$ in their domain. Also, for player $i$ to be indifferent between playing $b\in{[\underline{b},\overline{b}]}$, $\underline{b}$, and $\overline{b}$, it must be that his expected payoff is constant. Let $F_{i}(b)$ be the distribution with which player $i$ bids.

\textcolor{blue}{\textbf{Claim:} If $v_{i}=v_{j}=v$ then $\underline{b}=0$, $\overline{b}=v$}

\textbf{Proof}
\begin{itemize}
\item Let $\overline{b}>v$. Then the probability of winning is one and the expected payoff is negative which is less than expected payoff in the tie breaking rule.
\item Let $\overline{b}<v$. Then $E[u_{i}|]\overline{b}]=vF_{j}(\overline{b})-\overline{b}<E[u_{i}|\overline{b}+\epsilon]=v-\overline{b}-\epsilon$, for sufficiently small $\epsilon$.
\item Let $\underline{b}<0$. This contradicts bids being weakly positive.
\item Let $\underline{b}>0$. Then $E[u_{i}|\underline{b}]=-\underline{b}<E[u_{i}|0]=0$
\end{itemize}

Given this interval, below are the payoffs to bidding $b\in{[\underline{b},\overline{b}]}$, $\underline{b}$, and $\overline{b}$:

\begin{align}
E[u_{i}|b]&=vF_{j}(b)-b\nonumber\\
E[u_{i}|\overline{b}]&=v-\overline{b}=0\nonumber\\
E[u_{i}|\underline{b}]&=-\underline{b}=0\nonumber
\end{align}

So setting them equal to each other, we find that:

\begin{equation}
F_{j}(b)=\left\{
\begin{array}{ccc}
b/v& \mbox{if}& b\in[0,v]\\
0 & & o.w\\
\end{array}\right.
\end{equation}

\item Given this equilibrium CDF, the expected revenue for the seller is:

\begin{equation}
\nonumber
E[R]=2E[b]=2\int_{0}^{v} \frac{b}{v}db=v
\end{equation}
\end{enumerate}

\section{Problem 2}

\begin{enumerate}
\item To find the equilibrium prices in the second stage we will consider different regions.
\begin{itemize}
\item \textcolor{blue}{\textbf{Claim}: if $(k_{1}, k_{2})\geq(1,1)$, then $p_{1}^{*}=p_{2}^{*}=0$}

\textbf{Proof}: Suppose $k_{1}<1$ and $p_{1}=0$. Then firm 2 faces residual demand $1-k_{1}$, and in maximizing profits it will set $p_{2}=1$ making profits of $\pi_{2}=1-k_{1}$ is unconstrained or $\pi_{2}=k_{2}$ if constrained. Both these profits are greater than zero which is the profit it will get by setting $p_{2}=0$. Hence marginal cost pricing is not equilibrium when both $k_{1}$ and $k_{2}$ are strictly less than 1.

\item \textcolor{blue}{\textbf{Claim}: if $(k_{1}, k_{2})<(1,1)$ and $k_{1}+k_{2}\leq1$, then $p_{1}^{*}=p_{2}^{*}=1$}

\textbf{Proof}: The profits for firm 1 when setting $p_{1}=1$ are $\pi_{1}=k_{1}$. This firm does not have incentives to decrease its price since it cannot sell more than $k_{1}$. If it increases the price, then it sells zero and makes zero profit, which is worse. 

\item Suppose $(k_{1},k_{2})$ are not in the regions considered above.
\begin{itemize}
\item Let $k_{1}<1$ and $k_{2}>1$. Because firm 2 can serve the entire market it will set $p_{2}=0$, therefore firm 1 is undersold and faces a negative residual demand which is a contradiction. 
\item Let $(k_{1}, k_{2})<(1,1)$ and $k_{1}+k_{2}>1$. Assume without loss of generality that $k_{2}>k_{1}$ then:
\begin{itemize}
\item If firm 1 sets $p_{1}<1$, then $\pi_{1}=p_{1}k_{1}$ and firm 2 faces residual demand $1-k_{1}$ so it will optimize by setting $p_{2}=1$, in which case firm 1 would want to deviate.
\item If firm 1 sets $p_{1}>1$, then $\pi_{1}=0$ and the market breaks down since supply wouldn't meet demand.
\item If firm 1 sets $p_{1}=1$, then if $p_{2}=1$, they split the market and with our tie breaking rule firm i's profits are $\pi_{i}=k_{i}/(k_{i}+k_{j})$ but if $p_{2}=1-\epsilon$, then $\pi_{2}=(1-\epsilon)k_{2}>k_{i}/(k_{i}+k_{j})$ for $\epsilon$ sufficiently small. So, firm 2's best response is to set $p_{2}=1-\epsilon$, in which case   firm 1 sets $p_{1}=p_{2}-\epsilon$. Therefore, this is not a Nash equilibrium.
\end{itemize}
\end{itemize}
\end{itemize}

\item Using the result from lemma 2 in the lecture notes, let $G_{j}(p)=Pr(p_{j}<p)$ be the CDF of prices defined on the domain ${[\underline{p}, \overline{p}]}$. The expected payoff for firm $i$ in the mixed strategy equilibrium for the remaining set of capacity choices is:

\begin{equation}
\pi_{i}(p,G_{j})=G_{j}(p)p\min\{k_{i}, \max\{0,1-k_{j}\}\}+[1-G_{j}(p)]p\min\{k_{i},1\}
\end{equation}

We know that the firm with larger capacity will set higher prices. In order to maximize payoffs, the highest price it can charge is 1, therefore $\overline{p}=1$. Also, for firm $i$ to be indifferent between setting $p\in{[\underline{p},\overline{p}]}$, $\underline{p}$, and $\overline{p}$, it must be that its expected payoff is constant. For the boundaries of the interval, the payoff is presented below:

\begin{align}
\pi_{i}(\overline{p},G_{j})&=\pi_{i}(1,G_{j})=\min\{k_{i}, \max\{0,1-k_{j}\}\}\nonumber\\
\pi_{i}(\underline{p},G_{j})&=\underline{p}\min\{k_{i},1\}\nonumber
\end{align}

Setting these two expressions equal to each other we find:

\textcolor{blue}{\begin{equation}
\underline{p}=\frac{\min\{k_{i}, \max\{0,1-k_{j}\}\}}{\min\{k_{i},1\}}
\end{equation}}

And we can solve for $G_{j}(p)$ by setting $\pi_{i}(p,G_{j})=\pi_{i}(\overline{p},G_{j})$, so:

\textcolor{blue}{\begin{equation}
G^{*}_{j}(p)=\frac{\min\{k_{i}, \max\{0,1-k_{j}\}\}-p\min\{k_{i},1\}}{p[\min\{k_{i}, \max\{0,1-k_{j}\}\}-\min\{k_{i},1\}]}
\end{equation}}

To prove these are actually CDFs, assume $k_{i}<1, k_{j}>1$, then we want to show that for $G_{i}^{*}(p)$:  $G_{i}^{*}(\overline{p})=1$, $G_{i}^{*}(\underline{p})=0$, and $G_{i}^{*}(p)$ is strictly increasing. And for $G_{j}^{*}(p)$, we want to show $G_{j}^{*}(\underline{p})=0$ and that it has a mass point at $\overline{p}$. 

\textbf{Proof}
\begin{itemize}
\item
\begin{align}
G_{i}^{*}(\overline{p})&=\frac{1-k_{i}-\overline{p}}{\overline{p}[1-k_{i}-1]}=1\nonumber\\
G_{i}^{*}(\underline{p})&=\frac{1-k_{i}-\underline{p}}{\underline{p}[1-k_{i}-1]}=0\nonumber
\end{align}
\begin{align}
\frac{\partial G_{i}^{*}(p)}{\partial p}&=\frac{\min\{k_{j},1\}\min\{k_{j}, \max\{0,1-k_{i}\}\}-\min^{2}\{k_{j}, \max\{0,1-k_{i}\}\}}{p^{2}[\min\{k_{j}, \max\{0,1-k_{i}\}\}-\min\{k_{j},1\}]^{2}}\nonumber\\
&=\frac{(1-k_{1})-(1-k_{i})^{2}}{[\cdot]^{2}}>0\nonumber
\end{align}

\item For $G_{j}$, notice that $\pi_{i}(\overline{p})=0<\pi_{i}(\underline{p})=1-k_{i}$ so $\lim_{p\to\overline{p}}G_{j}(p)<1$ determining the mass point and $G_{j}^{*}(\underline{p})=0$.
\end{itemize}

\item  Now we move to the first stage to find the subgame perfect equilibrium in the full game and derive firm profits. 

\begin{itemize}
\item Given $p_{1}^{*}=p_{2}^{*}=0$ and $(k_{1}, k_{2})\geq(1,1)$, we have that first stage profits are $\pi_{i}=-ck_{i}$. Then firm $i$ has incentives to reduce its installed capacity. So there cannot be a Nash equilibrium in capacities in this region.
\item \textcolor{blue}{Given $p_{1}^{*}=p_{2}^{*}=1$ with $(k_{1}, k_{2})<(1,1)$ and $k_{1}+k_{2}\leq1$, we have that first stage profits are $\pi_{i}=(1-c)k_{i}$, so firm $i$ would like to set $k_{i}$ as high as possible given $k_{1}+k_{2}\leq1$. Therefore, $k_{1}+k_{2}=1$ and all of the points in this line are going to be a Nash equilibrium in the first stage.}
\item Given the Nash equilibrium in mixed strategies $[G_{1}^{*}(p), G_{2}^{*}(p)]$ for capacity choices that are not in the above regions, firm $i$'s profits are
 \begin{align}
\pi_{i}(p,G^{*}_{j})=G^{*}_{j}(p)p\min\{k_{i},\max\{0,1-k_{j}\}\}+[1-G^{*}_{j}(p)]p\min\{k_{i},1\}-ck_{i}
\end{align}
\begin{itemize}
\item Consider the case where $k_{1}<1$ and $k_{2}>1$. Then, the CDFs become:
\begin{align}
G_{1}^{*}(p)&=\frac{1-k_{1}-p}{-pk_{1}}\nonumber\\
G_{2}^{*}(p)&=1\nonumber
\end{align}
and profits for firm 1 are 
\begin{equation}
\nonumber
\pi_{1}=-ck_{1}
\end{equation}
Therefore, $k_{1}=0$, in which case firm 2 strictly prefers $p_{2}=1$. This contradicts the fact that both firms are strictly randomizing and we can not have a Nash equilibrium in the first stage for this region of capacity choices. By symmetry we can also rule out the region where $k_{1}>1$ and $k_{2}<1$.

\item Consider the case where $(k_{1},k_{2})<(1,1)$ and $k_{1}+k_{2}>1$. Then the CDFs become:
\begin{align}
G_{1}^{*}(p)&=\frac{1-k_{1}-pk_{2}}{p[1-k_{1}-k_{2}]}\nonumber\\
G_{2}^{*}(p)&=\frac{1-k_{2}-pk_{1}}{p[1-k_{2}-k_{1}]}\nonumber
\end{align}
and profits for firm 1 are 
\begin{align}
\pi_{1}&=\left(\frac{1-k_{1}-pk_{2}}{1-k_{1}-k_{2}}\right) [1-k_{2}]+\left(1- \frac{1-k_{1}-pk_{2}}{p[1-k_{1}-k_{2}]}\right)pk_{1}-ck_{1}\\
&=1-k_{1}-pk_{2}+pk_{1}-ck_{1}\\
&=1-pk_{2}+(p-c-1)k_{1}
\end{align}
So once again, if firm 1 maximizes first stage profits it will set $k_{1}=0$, which contradicts the original assumptions. 
\end{itemize}
\end{itemize}

Summarizing, the SPE on this game is $k_{i}+k_{j}=1$ and $p_{i}=p_{j}=1$ with profits given by: $\pi_{i}=(1-c)k_{i}$

\end{enumerate}

\end{document}
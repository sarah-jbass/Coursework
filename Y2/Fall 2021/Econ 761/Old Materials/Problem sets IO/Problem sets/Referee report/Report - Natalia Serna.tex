\documentclass[12pt,english]{article}
\usepackage[T1]{fontenc}
\usepackage{babel}
\usepackage[margin=0.8 in]{geometry}
\usepackage{caption}
\usepackage{subfig}
\usepackage{longtable}
\usepackage{natbib}
\linespread{1.15}
\usepackage{tikz}
\usepackage{setspace}
\usepackage{multirow}
\usepackage{multicol}
\usepackage{csquotes}
%\usepackage{mathds}
% ------
% Fonts and typesetting settings


%\usepackage[sc]{mathpazo}
\usepackage{titling}									
\usepackage{float}
\usepackage{pdflscape}
\usepackage[toc]{appendix}
\renewcommand{\appendixtocname}{Appendices}
%\renewcommand{\appendixsection}{\normalfont\bfseries}
\usepackage{amsmath, amsthm, amsfonts}


\newcommand{\subtitle}[1]{%
  \posttitle{%
    \par\end{center}
    \begin{center}\large#1\end{center}
    \vskip0.5em}%
}

\usepackage{booktabs}												
\usepackage{natbib}                                                 
\usepackage{graphics,epsfig}						
% -----
% ------

% Node styles
\tikzset{
% Two node styles for game trees: solid and hollow
solid node/.style={circle,draw,inner sep=1.5,fill=black},
hollow node/.style={circle,draw,inner sep=1.5}
}


% Maketitle metadata
\author{
Natalia Serna
   }
 \title{Mergers When Prices Are Negotiated: Evidence from the Hospital Industry (Gowrisankaran, Nevo, Town, 2015)}
 \subtitle{Referee report}

\date{}
%%%%%%%%%%%%%%%%%%%%%%%%
\begin{document}

\maketitle


\section{Overview}

This paper estimates a bargaining model of competition between hospitals and MCOs to study the effect of horizontal mergers in the upstream market (hospitals). The model assumes MCOs incentives are perfectly aligned to those of patients, in the sense they maximize patient expected utility from the hospital network that the MCO bargains with, while hospitals maximize profits. There are two stages in the bargaining game: in the first stage, hospital systems and MCOs bargain over the base price that each hospital will be paid by each MCO for hospital care and, in the second stage, patients receive a health shock and choose a hospital within the network of its MCO to maximize their utility, which is a function of the out-of-pocket expense. This out-of-pocket expense is one of the main variables in this paper: it is the negotiated base price multiplied by the coinsurance rate and a resource intensity of the illness. Low coinsurance rates, which are in fact observed in the data, will be associated to lower demand elasticities for a particular hospital and this motivates the bargaining model over the usual Bertrand-Nash price setting. The reason is that with inelastic demands as in this context, the Bertrand-Nash solution for prices, which coincides with a bargaining model where all the bargaining power lies in hospitals, would yield negative estimates of marginal costs or, in other words, would not be consistent with profit maximization. As a result, the bargaining model allows to distinguish between the part of the matrix of derivatives of demand coming from hospital profit maximization and the part coming from MCO patient welfare maximization. Coinsurance rates also allow MCOs to partly steer patients toward cheaper hospitals, so when analyzing mergers, this implicit competition between hospitals in a network, will result in a post-merger increase in prices that is lower under a bargaining model than under the Bertrand-Nash model.

One particular feature of the bargaining model is that to find the equilibrium base prices for one MCO-hospital system pair, the negotiated prices of this MCO with all other hospitals systems are assumed fixed. This, in addition to the fact that patients choose hospitals and not plans implies that, under disagreement, a hospital system cannot capture back any of the MCOs patients through plan switches.

The authors use the estimates of the bargaining model to evaluate the welfare effects of the Inova/Prince William merger in Northern Virginia and find that negotiated base prices go up but patient volume in the merged system goes down only slightly due to the low coinsurance rates. The authors also perform several robustness exercises: one that captures the extent to which the moral hazard problem (that leads to overconsumption of health services by insured individuals) is resolved by setting coinsurance rates, and another one that allows for MCOs objective function to be profits instead of patient welfare. 

This paper contributes to the literature of empirical models of bargaining and Nash-in-Nash solution concept as well as to the literature on vertical relations and healthcare markets. 

\section{Model}

The authors model competition between insurers and MCOs using a two stage game. In the first stage, MCOs and hospital systems bargain over the base price for health services, and in the second stage patients receive a health shock and choose a hospital in order to maximize their utility. As usual, the model is solved backwards starting from the patient's choice of hospital. Note that the choice of MCO is fixed, this is one of the main differences with other papers addressing bargaining in health care markets like Ho and Lee. 

With certain probability a patient can get a disease $d$. This disease is associated to a relative consumption of health services denoted by $w_{d}$. The patient's MCO pays the hospital the negotiated base price, which is taken as given in the second stage, times $w_{d}$. However, the actual price paid by the patient, or the out-of-pocket expense, is the negotiated base price, times $w_{d}$, times the coinsurance rate $c_{id}$. The out-of-pocket expense enters patient $i$'s indirect utility function for hospital $j$, together with hospital and patient characteristics. The patient-hospital specific error term in the indirect utility function is assumed to be distributed extreme value type I, which generates the closed from logit expressions for patient's choice probabilities. In this case, the outside option is defined as treatment at a hospital outside the geographic area. This definition can be problematic as will be argued ahead in this report. Using the choice probabilities, we can derive the ex ante expected utility for the hospitals in the network of an MCO. In particular, we can derive the ``willingness-to-pay'' (WTP)  for hospital $j$ as the difference in the ex ante expected utility of the MCO's current hospital network and the ex ante expected utility of the hospital network excluding $j$. The sum across all hospitals in the network of these WTP (weighted by the relative importance of employee welfare, $\tau$) minus total costs (demand for hospital $j$ times base price, times $w_{d}$, times $1-c_{id}$) will be the MCO's objective function. The hospital's profit function, on the other hand, is  equal to the negotiated base price minus marginal cost times demand for the hospital. The authors assume hospital marginal costs are constant in quantity and have a unobserved component to the econometrician. They also assume there are no capacity constraints so that patient switching is always possible. 

Then we move on to the first stage of the game where hospital systems and MCO's bargain over base prices. The main critical but arguable assumption at this stage is that MCOs and hospitals have complete information about hospital and enrollee attributes. Another main assumption common to the literature of Nash-in-Nash bargaining setups is that each contract remains the same even if negotiation for another contract fails. Each MCO-hospital system pair maximizes the Nash surplus weighted by their relative bargaining powers. The system of first order conditions can then be rewritten in matrix form so that it ends up looking very familiar to the BLP model but where the matrix of derivatives of demand with respect to price account for both the variations in hospital surplus and the variation in patient welfare coming the MCO's objective function.

Replacing the expression for hospital marginal costs in the FOC of the Nash surplus maximization problem, we can solve for the unobserved component of marginal costs, which the authors define as the econometric error. This error interacted with instruments will yield the moment conditions for GMM estimation.

\section{Data}

To estimate the bargaining model and study the effects of horizontal mergers in the upstream hospital market, the authors use claims data from four large MCOs in Northern Virginia and inpatient discharge data from Virginia Health Information, form 2003 to 2006. For every patient encounter, they observe: demographic characteristics, diagnosis, procedure, DRG, and the actual amount paid to the hospital for each claim. Because there are multiple claims per inpatient stay, data is aggregated to the inpatient episode level. The base price is the fitted value of a regression of total amount paid divided by DRG weight on gender, age, and hospital dummies, estimated separately for each MCO-year.

The inpatient discharge data from the Virginia Health Information is the one used to estimate the second stage of the model. The authors use the subsample of patients in acute care excluding those over 64 years of age and newborns.

\section{Analysis}

The authors use the model to study the competitive interactions between hospitals and MCOs in Northern Virginia and analyze the impact of the merger between Inova Health System and Prince William Hospital. First they show some summary statistics that evidence the significant variation in prices and market shares (measured as shares by discharges) across hospitals prior to the merger as well as significant differences in the patients they serve, for example, there is high variation in their DRG weights and travel times. These summary statistics evidence the considerable hospital heterogeneity.

Results from the estimation of the second stage of patient hospital choice, performed using maximum likelihood, show that patients are sensitive to travel times and out-of-pocket price. The implied elasticities from this model suggest demand for hospitals is inelastic, which would be problematic under a Bertrand competition assumption.

For the first stage bargaining game, the authors estimate two specifications: the first one, where they fix the bargaining weights of each party to a half and, the second one, where they allow the bargaining power to vary across MCOs. In the first specification, the authors find that MCOs place twice as much weight on enrollee welfare as on reimbursed costs in their objective function and that there is large variation in the hospital marginal costs across MCOs. In the second specification, findings suggest there is significant heterogeneity in bargaining power across MCOs and, in general, MCOs have more leverage than hospitals. Although both specifications allow for differences in WTP to explain variation in prices, it is the argument of differences in hospital costs and bargaining power that fits the data better. The authors conclude that the first specification is the most salient both because of interpretation reasons (the bargaining parameter can be interpreted as the agent's discount factors) and because results from the second specification indicating prices for some MCOs are equal to their reservation values, would imply unreasonable variations in prices even after a monopoly of hospitals. Results also show there are significant differences between the actual price elasticity (the one that is consistent with all the bargaining power being in the hands of hospitals) and the effective price elasticity, with the latter being larger than the former, and implying positive estimates of marginal costs, consistent with the theory.

The authors conduct four counterfactual analysis: (i) the impact of the Inova/PWH merger using the bargaining parameter implied by the model, (ii) the impact of the Inova/PWH merger with different bargaining powers, (iii) the demerger of Loudoun Hospital from Inova, (iv) and breaking up the Inova system. For counterfactual 1, findings show the merger leads to a significant increase in prices but no significant variation in the volume of patients, which reflects low coinsurance rates and the fact that in the new equilibrium rival hospitals also increase their prices. In counterfactual 2, the assumption is that after Inova acquires PWH, all the hospitals in the new Inova system, including PWH, must negotiate separately with each MCO. In this case, if PWH fails to reach an agreement with an MCO, this does not mean that the entire Inova system is excluded from that MCO's network, but only  PWH. Theoretically, results from this counterfactual are ambiguous because both parties' disagreement payoffs increase with the proposed policy. Empirically, the authors find very similar results to those in counterfactual 1 suggesting the proposed policy has no effect in prices but these will go up regardless after the merger. In the third counterfactual, results are opposite in nature to those from counterfactual 1. After divesting from the Loudoun hospital, competition between hospitals increase, which increases the disagreement payoffs to the MCOSs, and thus decreases the equilibrium prices. Results from the fourth counterfactual are similar in nature to those of the third counterfactual, in the sense that breaking up the Inova system, equilibrium prices would decrease by almost 7 percent market-wide.

Their last results relate to robustness exercises in which the MCO's objective function is now profit. This implies changing the first stage of the game by allowing MCOs to simultaneously post premiums after bargaining base prices with hospitals, and compete for enrollees. The second stage where patients receive a health shock remains the same. Estimation of this model indicates that base prices are slightly lower than in the base model and per-patient margins (relative to the premium) are slightly higher. Regarding counterfactuals, this alternative specification predicts a higher price increase after the Inova/PWH merger compared to the base model.

\section{Critique}

I have several critiques to this paper, outlined below:
\begin{itemize}
\item The authors define the outside option as receiving treatment at a hospital located outside the geographic area and define the utility from choosing this option only as a function of the out-of-pocket expense and an unobservable term. However, by the way the authors choose the subsample of patients for estimation, i.e, those who require acute care, we can argue that there really isn't an outside option, but instead patients have to receive treatment in an hospital within the geographic area given that moving outside of it is not viable for these sick enough patients.
\item In the second stage of the model where patients receive a health shock, the choice of hospital where to receive treatment is conditioned on the negotiated base price of the hospital system through the out-of-pocket expense but in the first stage these base prices are allowed to vary within a hospital system. The authors do not explain how do they aggregate from individual hospital base prices to hospital system base prices, and this seems an important source of variation in the data that could help identify the parameters in the utility function. 
\item One of the main assumptions in the bargaining stage of the model is that both hospitals and MCOs have complete information about the patient's health status and hospital costs. This is a strong assumption which is difficult to relax. The authors should disclose that although not necessarily true, the assumption simplifies the model and keeps it tractable. In reality, hospitals have private information about patient's health status and may have incentives to overreport service consumption to the MCOs. Even though this is beyond the scope of the paper, it would help the reader understand where possible extensions to the model could lie on.
\item The authors conclude in favor of the model where MCOs maximize the weighted sum of patient welfare and costs over the model where MCOs maximize profits,  based only on the implied results of a counterfactual world where there is a monopoly of hospitals. I believe the model selection argument should be based on which fits the data better. Although the second specification yields unreasonable price increases after a monopoly of hospitals, this can be explained by the fact that competition between MCOs which does not exist in the first specification may be driving price increases down. In other words, the bargaining model with competition both in the upstream and downstream markets is not necessarily wrong, but the way it is implemented could be the reason why price increases are underestimated. For instance, instead of  imposing the structural parameter obtained from the first specification, this second model may require reestimating the full set of parameters.
\item A more detailed explanation of the choice of instruments is missing from the paper.
\item A more detailed explanation of how computation of counterfactuals work is missing from the paper.
\item The predicted increase in prices following the Inova/PWH merge in  counterfactual 1 may be overestimated. To understand why, first note that post-merger equilibrium prices can go up both because of the increased leverage of the merged hospital as well as because of the increase in equilibrium prices by rival hospitals. The latter effect could be lower in the case hospitals are specialized in the treatment of different diseases. This appears to be the case after analyzing the tables of summary statistics where the authors show that there is significant variation in the DRG weights across hospitals. This may be an indication of the need for better instruments in estimation.
\end{itemize}

\section{Evaluation: Revise and resubmit}

The paper by Gowrisankaran, Nevo and Town addresses the question of how hospital mergers impact welfare in health markets where prices are negotiated. The authors use a structural bargaining model of competition between hospitals, where in the first stage hospital systems and MCOs bargain over health services' prices and in the second stage patients choose hospitals based on their health status. The motivation to use a bargaining model instead of a usual Bertrand competition model is very clear from the paper as well as the gains from the  identification strategy for studying the impact of mergers in equilibrium prices and volume of patients. Although the paper is easy to follow, there are several issues that need to be addressed. Summarizing, these critiques relate to: 

\textbf{Major comments}
\begin{itemize}
\item Discussing the relevance and exogeneity of their instruments.
\item Explaining how negotiated base prices from the first stage -which are allowed to vary within a hospital system- are aggregated before entering the patient's utility function in the second stage as just a hospital system-specific base price.
\end{itemize}

\textbf{Minor comments}
\begin{itemize}
\item Improving the model in their robustness check exercise before selecting the specification that fits the data better.
\item Discussing the implications of defining the outside option as receiving treatment in a hospital outside the geographic area for the subsample of patients they focus on (acute care).
\end{itemize}
The main contribution of this paper to the literature of empirical industrial organization is the application of the Nash-in-Nash framework to model the impact of mergers. Other papers in this same area are more concerned about the best way to model disagreement payoffs as in Ho and Lee (2018) and network formation as in Ho (2009). In general, the research question is interesting not only in the context of the US health care system. The methodology used to answer this question is appropriate and the analysis convincing. 

\end{document}
\documentclass[12pt,english]{article}
\usepackage[T1]{fontenc}
\usepackage{babel}
\usepackage[margin=0.8 in]{geometry}
\usepackage{caption}
\usepackage{subfig}
\usepackage{longtable}
\usepackage{natbib}
\linespread{1.15}
\usepackage{tikz}
\usepackage{setspace}
\usepackage{multirow}
\usepackage{multicol}
\usepackage{csquotes}
% ------
% Fonts and typesetting settings


%\usepackage[sc]{mathpazo}
\usepackage{titling}									
\usepackage{float}
\usepackage{pdflscape}
\usepackage[toc]{appendix}
\renewcommand{\appendixtocname}{Appendices}
%\renewcommand{\appendixsection}{\normalfont\bfseries}
\usepackage{amsmath, amsthm, amsfonts}


\newcommand{\subtitle}[1]{%
  \posttitle{%
    \par\end{center}
    \begin{center}\large#1\end{center}
    \vskip0.5em}%
}

\usepackage{booktabs}												
\usepackage{natbib}                                                 
\usepackage{graphics,epsfig}						
% -----
% ------



% Maketitle metadata
\author{
Natalia Serna
   }
 \title{Problem set 3}

\date{}
%%%%%%%%%%%%%%%%%%%%%%%%
\begin{document}

\maketitle


\begin{enumerate}
\item 
\begin{enumerate}
\item The marginal consumer between Starbucks-0 and Esquire, $\hat{x}_{0}$ is such that:
\begin{align}
u(\hat{x}_{0})=3-p_{0}-t\hat{x}_{0}^2&=3-q-t(1/2-\hat{x}_{0})^{2}=u(1/2-\hat{x}_{0})\nonumber\\
p_{0}+t\hat{x}_{0}^{2}&=t/4-t\hat{x}_{0}+t\hat{x}_{0}^{2}+q\nonumber
\end{align}
\textcolor{blue}{
\[
\hat{x}_{0}&=\frac{1}{4}+\frac{q}{t}-\frac{p_{0}}{t}
\]
}
The marginal consumer between Starbucks-1 and Esquire, $\hat{x}_{1}$ is such that:
\begin{align}
u(\hat{x}_{1})=3-q-t(\hat{x}_{1}^{2}-1/2)^{2}&=3-p_{1}-t(1-\hat{x}_{1})^{2}=u(1-\hat{x}_{1})\nonumber\\
q+t\hat{x}_{1}^{2}-t\hat{x}_{1}+t/4&=p_{1}+t-2t\hat{x}_{1}+t\hat{x}_{1}^{2}\nonumber
\end{align}
\textcolor{blue}{
\[
\hat{x}_{1}&=\frac{3}{4}+\frac{p_{1}}{t}-\frac{q}{t}
\]
}
\item The demand for Starbucks-0 is:
\[
Q_{0}&=\frac{1}{4}+\frac{q}{t}-\frac{p_{0}}{t}
\]

The demand for Esquire is:
\[
Q_{e}=\hat{x}_{1}-\hat{x}_{0}=\frac{1}{2}+\frac{p_{1}+p_{0}}{t}-\frac{2q}{t}
\]

The demand for Starbucks-1 is:
\[
Q_{e}=1-\hat{x}_{1}=\frac{1}{4}+\frac{q}{t}-\frac{p_{1}}{t}
\]

Therefore, Starbucks solves the problem of
\begin{align}
\max_{p_{0},p_{1}}& \left(\frac{1}{4}+\frac{q}{t}-\frac{p_{0}}{t} \right)p_{0} +\left( \frac{1}{4}+\frac{q}{t}-\frac{p_{1}}{t}\right)p_{1}\nonumber\\
\mbox{FOC wrt $p_{0}$:}\ \ & \textcolor{blue}{p_{0}=\frac{t}{8}+\frac{q}{2}}\nonumber\\
\mbox{FOC wrt $p_{1}$:}\ \ & \textcolor{blue}{p_{1}=\frac{t}{8}+\frac{q}{2}}\nonumber
\end{align}

Esquire, on the other hand solves:
\begin{align}
\max_{q}& \left(\frac{1}{2}+\frac{p_{1}+p_{0}}{t}-\frac{2q}{t}\right)q\nonumber\\
\mbox{FOC:}\ \ & \textcolor{blue}{q=\frac{t}{8}+\frac{p_{0}+p_{1}}{4}}\nonumber
\end{align}

\item Solving for the equilibrium:
\begin{align}
\left(\begin{array}{ccccc}
1&0&-1/2&|&t/8\\
0&1&-1/2&|&t/8\\
1/4&1/4&-1&|&-t/8
\end{array}\right)\nonumber
\end{align}

Reducing this matrix yields:
\textcolor{blue}{
\[
p_{0}=p_{1}=q=t/4
\]
}

\item In this case, the demand for Esquire is:
\[
Q_{e}&=\frac{1}{4}+\frac{p_{0}}{t}-\frac{q}{t}
\]

The demand for Starbucks-0 is:
\[
Q_{0}=\hat{x}_{1}-\hat{x}_{0}=\frac{1}{2}+\frac{p_{1}+q}{t}-\frac{2p_{0}}{t}
\]

And the demand for Starbucks-1 is:
\[
Q_{1}=1-\hat{x}_{1}=\frac{1}{4}+\frac{p_{0}}{t}-\frac{p_{1}}{t}
\]

So, Esquire solves the following problem

\begin{align}
\max_{q}& \left(\frac{1}{4}+\frac{p_{0}}{t}-\frac{q}{t}\right)q\nonumber\\
\mbox{FOC:}\ \ & \textcolor{blue}{q=\frac{t}{8}+\frac{p_{0}}{2}}\nonumber
\end{align}

And Starbucks solves the following problem:
\begin{align}
\max_{p_{0},p_{1}}& \left(\frac{1}{2}+\frac{p_{1}+q}{t}-\frac{2p_{0}}{t}\right)p_{0} +\left(\frac{1}{4}+\frac{p_{0}}{t}-\frac{p_{1}}{t}\right)p_{1}\nonumber\\
\mbox{FOC wrt $p_{0}$:}\ \ & \textcolor{blue}{p_{0}=\frac{t}{8}+\frac{p_{1}}{2}+\frac{q}{4}}\nonumber\\
\mbox{FOC wrt $p_{1}$:}\ \ & \textcolor{blue}{p_{1}=\frac{t}{8}+p_{0}}\nonumber
\end{align}

Solving for the equilibrium:
\begin{align}
\left(\begin{array}{ccccc}
-1/2&0&1&|&t/8\\
1&-1/2&-1/4&|&t/8\\
-1&1&0&|&t/8
\end{array}\right)\nonumber
\end{align}

Reducing this matrix yields:
\textcolor{blue}{
\[
p_{0}=7t/12, \ \ p_{1}=17t/24,\ \ q=5t/12
\]
}

\item The reason why these two frameworks yield different equilibria in prices is because in the first case, Starbucks'  stores are not competing with each other,  so the only strategic substitute is Esquire for both of them. This allows Starbucks to charge the same price in both stores. While in the second case, Starbucks' stores compete against each other, so the strategic substitute is not only esquire but the other store as well. In this case, because the store located in the middle faces competition from Esquire and Starbucks itself, it has to set a price that is lower than the price set by the Starbucks' store located at the end of the unit line. Also because competition for Esquire is lower when it locates at the end it can charge a higher price compared to the situation where it is located in the middle.

\end{enumerate}

\item
\begin{enumerate}
\item A consumer is indifferent between buying s=1 and s=2 if:
\[
2(\theta_{2}-p_{2})=\theta_{2}-p_{1}\to \ \theta_{2}=2p_{2}-p_{1}
\]
A consumer is indifferent between buying s=1 and the outside option if:
\[
\theta_{1}-p_{1}=0\to \ \theta_{1}=p_{1}
\]
Therefore, demand for s=2 is:
\[
Q_{2}=Prob(\theta\geq\theta_{2})=1-2p_{2}+p_{1}
\]
Demand for s=1 is:
\[
Q_{1}=Prob(\theta_{1}\leq\theta\leq \theta_{2})=2(p_{2}-p_{1})
\]
The monopolist producing both qualities solves the following problem:
\begin{align}
\max_{p_{1},p_{2}}& 2(p_{2}-p_{1})(p_{1}-c)+(1-2p_{2}+p_{1})(p_{2}-2c)\nonumber\\
\mbox{FOC wrt $p_{1}$:}\ \ & \textcolor{blue}{p_{1}=\frac{3p_{2}}{4}}\nonumber\\
\mbox{FOC wrt $p_{2}$:}\ \ & \textcolor{blue}{p_{2}=\frac{3p_{1}+2c+1}{4}}\nonumber
\end{align}
So the solution to this system of linear equations yields:
\textcolor{blue}{
\[
p_{1}=\frac{3}{7}(2c+1),\ \ p_{2}=\frac{4}{7}(2c+1)
\]
}
And equilibrium profits are:
\textcolor{blue}{
\[
\pi_{2}=\frac{4}{7}c^{2}-\frac{8}{7}c+\frac{2}{7}
\]
}
\item If the monopolist only offers low quality, then it solves the problem:
\begin{align}
\max_{p_{1}}& (1-p_{1})(p_{1}-c)\nonumber\\
\mbox{FOC:}\ \ & \textcolor{blue}{p_{1}=\frac{1+c}{2}}\nonumber
\end{align}
And the equilibrium profits are:
\textcolor{blue}{
\[
\pi_{1}=\frac{(1-c)^{2}}{4}
\]
}
\item This suggests that offering both products is optimal whenever $\pi_{2}\geq\pi_{1}$ or:
\[
\frac{4}{7}c^{2}-\frac{8}{7}c+\frac{2}{7}\geq\frac{(1-c)^{2}}{4}\iff\ \ \textcolor{blue}{c\geq\frac{1+2\sqrt{2}}{3}}
\]
\item Now, suppose we have two firms competing, and demand for each product is weakly positive. From this condition we can derive the relevant range for $p_{2}$. So we have:

\begin{align}
Q_{1}=2(p_{2}-p_{1})\geq0&\iff p_{2}\geq p_{1}\nonumber\\
Q_{2}=1-2p_{2}+p_{1}\geq0&\iff p_{2}\leq (1+p_{1})/2\nonumber\\
\textcolor{blue}{p_{1}\leq p_{2}\leq(1+p_{1})/2}\nonumber
\end{align}

Given the relevant range for $p_{2}$, demand for firm 1 is:
\textcolor{blue}{
\[
Q_{1}=
\left\{\begin{array}{ccc}
1-p_{1}&\mbox{if}&p_{2}>(1+p_{1})/2\\
2(p_{2}-p_{1})&\mbox{if}&p_{1}\leq p_{2}\leq(1+p_{1})/2\\
0&\mbox{if}& p_{2}< p_{1}
\end{array}\right.
\]
}
What this is telling us, is that when $p_{2}$ exceeds $(1+p_{1})/2$, firm 1 is a monopolist in the market, while when $p_{2}< p_{1}$ every consumer will strictly prefer to buy the high quality good.

\item The best response function for firm 1 is\footnote{In the case $p_{2}\leq(1+p_{1})/2$, $p_{1}=arg\max 2(p_{2}-p_{1})(p_{1}-c)$}:
\textcolor{blue}{
\[
p_{1}(p_{2})=
\left\{\begin{array}{ccc}
(1+c)/2&\mbox{if}&p_{2}>(1+p_{1})/2\\
(p_{2}+c)/2&\mbox{if}&p_{1}\leq p_{2}\leq(1+p_{1})/2\\
0&\mbox{if}& p_{2}< p_{1}
\end{array}\right.
\]
}
\item Firm 2 is solving the problem of:
\begin{align}
\max_{p_{2}}& (1-2p_{2}+p_{1})(p_{2}-2c)\nonumber\\
\mbox{FOC:}\ \ & \textcolor{blue}{p_{2}=\frac{1+4c+p_{1}}{4}}\nonumber
\end{align}

 The graph of the best response functions is shown below:

\begin{center}
\begin{tikzpicture}[domain=0:3]
\draw[->] (-1,0) -- (3.5,0)
node[below right] {$p_{1}$};
\draw[->] (0,-1) -- (0,3.5)
node[left] {$p_{2}$};
\draw (1/3,0)  
node[fill=black, circle, scale=0.008cm, label={below:{\footnotesize{$c$}}}] ;
\draw (5/9,0)  
node[fill=black, circle, scale=0.008cm, label={[label distance=0.75cm]below:{\footnotesize{$\frac{1+2c}{3}$}}}] ;
\draw[->] (5/9,0)--(5/9,-3/4);
\draw (2/3,0)  
node[fill=black, circle, scale=0.008cm, label={below:{\footnotesize{$\frac{1+c}{2}$}}}] ;
\draw[dashed] (0,0) -- (3,3)
node[label={above right:{\footnotesize{$p_{2}=p_{1}$}}}];
\draw[dashed] (0,1/2) -- (3,2)
node[label={above right:{\footnotesize{$p_{2}=(1+p_{1})/2$}}}];
\draw[variable=\x, red, line width=0.5mm] plot ({\x}, {(7/12)+(\x/4)})
node[label={above down:{\textcolor{red}{\footnotesize{$p_{2}(p_{1})$}}}}];
\draw[blue, line width=0.5mm] (1/3,1/3)--(5/9,7/9); 
\draw[blue, dashed, line width=0.5mm] (5/9,7/9) --(2/3,7/8);
\draw[blue, line width=0.5mm] (2/3,7/8)--(2/3,3)
node[label={above right:{\textcolor{blue}{\footnotesize{$p_{1}(p_{2})$}}}}];
\draw[blue, line width=0.4mm] (1/3,0)--(1/3,1/3);
\draw (0,7/12)  
node[fill=black, circle, scale=0.008cm, label={left:{\footnotesize{$\frac{1+4c}{4}$}}}] ;
\end{tikzpicture}
\end{center}

Now let's consider the extreme cases where $c=0$ and $c=1/2$

\begin{center}
\begin{tikzpicture}[domain=0:3]
\draw(1.5,5)
node[label={below:{$c=0$}}];
\draw[->] (-1,0) -- (3.5,0)
node[below right] {$p_{1}$};
\draw[->] (0,-1) -- (0,3.5)
node[left] {$p_{2}$};
\draw (1/3,0)  
node[fill=black, circle, scale=0.008cm, label={below:{\footnotesize{$\frac{1}{3}$}}}] ;
\draw (1/2,0)  
node[fill=black, circle, scale=0.008cm, label={below:{\footnotesize{$\frac{1}{2}$}}}] ;
\draw[dashed] (0,0) -- (3,3)
node[label={above right:{\footnotesize{$p_{2}=p_{1}$}}}];
\draw[dashed] (0,1/2) -- (3,2)
node[label={above right:{\footnotesize{$p_{2}=(1+p_{1})/2$}}}];
\draw[variable=\x, red, line width=0.5mm] plot ({\x}, {(1+\x)/4})
node[label={above down:{\textcolor{red}{\footnotesize{$p_{2}(p_{1})$}}}}];
\draw[blue, line width=0.5mm] (0,0)--(1/3,2/3); 
\draw[dashed] (1/3,2/3) --(1/2,3/4);
\draw[blue, line width=0.5mm] (1/2,3/4)--(1/2,3)
node[label={above right:{\textcolor{blue}{\footnotesize{$p_{1}(p_{2})$}}}}];
%\draw[blue, line width=0.5mm] (0,1/3)--(0,0);
%\draw[blue, line width=0.4mm] (1/3,0)--(1/3,1/3);
%\draw[blue, dashed, line width=0.4mm] (1,1.5)--(0.75,1.25);
%\draw (0,3/4)  
%node[fill=black, circle, scale=0.008cm, label={left:{\footnotesize{$\frac{3}{4}$}}}] ;
\end{tikzpicture}
\end{center}


\begin{center}
\begin{tikzpicture}[domain=0:3]
\draw(1.5,5)
node[label={below:{$c=1/2$}}];
\draw[->] (-1,0) -- (3.5,0)
node[below right] {$p_{1}$};
\draw[->] (0,-1) -- (0,3.5)
node[left] {$p_{2}$};
\draw (1/2,0)  
node[fill=black, circle, scale=0.008cm, label={below:{\footnotesize{$\frac{1}{2}$}}}] ;
\draw (2/3,0)  
node[fill=black, circle, scale=0.008cm, label={below:{\footnotesize{$\frac{2}{3}$}}}] ;
\draw[dashed] (0,0) -- (3,3)
node[label={above right:{\footnotesize{$p_{2}=p_{1}$}}}];
\draw[dashed] (0,1/2) -- (3,2)
node[label={above right:{\footnotesize{$p_{2}=(1+p_{1})/2$}}}];
\draw[variable=\x, red, line width=0.5mm] plot ({\x}, {(3+\x)/4})
node[label={above down:{\textcolor{red}{\footnotesize{$p_{2}(p_{1})$}}}}];
\draw[blue, line width=0.5mm] (1/2,1/2)--(2/3,3/4); 
\draw[blue, dashed, line width=0.5mm] (2/3,3/4) --(3/4,7/8);
\draw[blue, line width=0.5mm] (3/4,7/8)--(3/4,3)
node[label={above right:{\textcolor{blue}{\footnotesize{$p_{1}(p_{2})$}}}}];
\draw[blue, line width=0.5mm] (1/2,0)--(1/2,1/2);
%\draw[blue, line width=0.4mm] (1/3,0)--(1/3,1/3);
%\draw[blue, dashed, line width=0.4mm] (1,1.5)--(0.75,1.25);
%\draw (0,3/4)  
%node[fill=black, circle, scale=0.008cm, label={left:{\footnotesize{$\frac{3}{4}$}}}] ;
\end{tikzpicture}
\end{center}


So the firms can coexist depending on the value of $c$. For $c$ low enough, the Nash equilibrium in prices is:

\textcolor{blue}{
\[
p_{1}=\frac{1}{7}+\frac{8}{7}c, \ \ p_{2}=\frac{2}{7}+\frac{9}{7}c
\]
}
While for c close to $1/2$ we have a monopoly of firm 1.


%Another way to interpret the ``relevant'' range for $p_{2}$ is $\textcolor{blue}{p_{2}\in[2c, (1+2c)/2]}$, where the upper bound is the monopoly price of firm 2. Given this relevant range, the demand for firm 1 would be:
%
%\textcolor{blue}{
%\[
%Q_{1}=
%\left\{\begin{array}{ccc}
%1-p_{1}&\mbox{if}&p_{2}>(1+2c)/2\\
%2(p_{2}-p_{1})&\mbox{if}&2c\leq p_{2}\leq(1+2c)/2\\
%&\mbox{if}& p_{2}< 2c
%\end{array}\right.
%\]
%}
%
%And the best response functions are given by:
%
%\textcolor{blue}{
%\begin{align}
%p_{1}(p_{2})&=
%\left\{\begin{array}{ccc}
%(1+c)/2&\mbox{if}&p_{2}>(1+2c)/2\\
%(p_{2}+c)/2&\mbox{if}&2c\leq p_{2}\leq(1+2c)/2\\
%c&\mbox{if}& p_{2}< 2c
%\end{array}\right.\nonumber\\
%p_{2}(p_{1})&=\frac{1+4c+p_{1}}{4}\nonumber
%\end{align}
%}
%
%So graphing for a general $c$ we get:
%
%\begin{center}
%\begin{tikzpicture}[domain=0:3]
%\draw[->] (-1,0) -- (3.5,0)
%node[below right] {$p_{1}$};
%\draw[->] (0,-1) -- (0,3.5)
%node[left] {$p_{2}$};
%\draw (1/3,0)  
%node[fill=black, circle, scale=0.008cm, label={below:{\footnotesize{$c$}}}] ;
%\draw (0.7,0)  
%node[fill=black, circle, scale=0.008cm, label={[label distance=0.75cm]below:{\footnotesize{$\frac{1+4c}{4}$}}}] ;
%\draw[->] (0.7,0)--(0.7,-0.75);
%\draw (1,0)  
%node[fill=black, circle, scale=0.008cm, label={below:{\footnotesize{$\frac{1+c}{2}$}}}] ;
%\draw[dashed] (3,0.9)--(0,0.9):
%\draw [->] (0,0.9)--(-0.4,1.15)
%node[label={left:{\footnotesize{$\frac{1+2c}{2}$}}}];
%\draw[dashed] (3,1/3)-- (0,1/3)
%node[fill=black, circle, scale=0.008cm, label={left:{\footnotesize{$c$}}}];;
%\draw[variable=\x, red, line width=0.5mm] plot ({\x}, {(2.5+\x)/4})
%node[label={above down:{\textcolor{red}{\footnotesize{$p_{2}(p_{1})$}}}}];
%\draw[blue, line width=0.5mm] (1/3,1/3)--(0.7,0.9); 
%\draw [blue, line width=0.5mm] (1/3,0) --(1/3,1/3);
%\draw[blue, line width=0.5mm] (1,0.9)--(1,3)
%node[label={above right:{\textcolor{blue}{\footnotesize{$p_{1}(p_{2})$}}}}];
%%\draw[blue, line width=0.5mm] (0,1/3)--(0,0);
%\draw[blue, line width=0.4mm] (1,0.9)--(0.7,0.9);
%\draw (0,0.7)  
%node[fill=black, circle, scale=0.008cm, label={left:{\footnotesize{$\frac{1+4c}{4}$}}}] ;
%\end{tikzpicture}
%\end{center}
%
%And focusing on the extreme cases where $c=0$ and $c=1/2$ we have:
%
%
%
%\begin{center}
%\begin{tikzpicture}[domain=0:3]
%\draw (3,3)
%node[label={below:{$c=0$}}];
%\draw[->]
% (-1,0) -- (3.5,0)
%node[below right] {$p_{1}$};
%\draw[->] (0,-1) -- (0,3.5)
%node[left] {$p_{2}$};
%\draw (1/4,0)  
%node[fill=black, circle, scale=0.008cm, label={below:{\footnotesize{$\frac{1}{4}$}}}] ;
%\draw (1/2,0)  
%node[fill=black, circle, scale=0.008cm, label={below:{\footnotesize{$\frac{1}{2}$}}}] ;
%\draw[dashed] (3,1/2)--(0,1/2):
%\draw (0,1/2)
%node[label={left:{\footnotesize{$\frac{1}{2}$}}}];
%\draw[variable=\x, red, line width=0.5mm] plot ({\x}, {(1+\x)/4})
%node[label={above down:{\textcolor{red}{\footnotesize{$p_{2}(p_{1})$}}}}];
%\draw[blue, line width=0.5mm] (0,0)--(1/4,1/2); 
%\draw [blue, line width=0.5mm] (1/2,1/2) --(1/2,3);
%\draw[blue, line width=0.5mm] (1/4,1/2)--(1/2,1/2)
%node[label={above right:{\textcolor{blue}{\footnotesize{$p_{1}(p_{2})$}}}}];
%%\draw[blue, line width=0.5mm] (0,1/3)--(0,0);
%\end{tikzpicture}
%\end{center}
%
%
%\begin{center}
%\begin{tikzpicture}[domain=0:3]
%\draw (3,3)
%node[label={below:{$c=1/2$}}];
%\draw[->]
% (-1,0) -- (3.5,0)
%node[below right] {$p_{1}$};
%\draw[->] (0,-1) -- (0,3.5)
%node[left] {$p_{2}$};
%\draw[dashed] (0,1/2)-- (3,1/2);
%\draw (1/2,0)  
%node[fill=black, circle, scale=0.008cm, label={below:{\footnotesize{$\frac{1}{2}$}}}] ;
%\draw (3/4,0)  
%node[fill=black, circle, scale=0.008cm, label={below:{\footnotesize{$\frac{3}{4}$}}}] ;
%\draw[dashed] (0,1)--(3,1):
%\draw[variable=\x, red, line width=0.5mm] plot ({\x}, {(1.5+\x)/4})
%node[label={above down:{\textcolor{red}{\footnotesize{$p_{2}(p_{1})$}}}}];
%\draw[blue, line width=0.5mm] (1/2,0)--(1/2,1/2); 
%\draw [blue, line width=0.5mm] (1/2,1/2) --(3/4,1);
%\draw[blue, line width=0.5mm] (3/4,1)--(3/4,3)
%node[label={above right:{\textcolor{blue}{\footnotesize{$p_{1}(p_{2})$}}}}];
%%\draw[blue, line width=0.5mm] (0,1/3)--(0,0);
%\end{tikzpicture}
%\end{center}
%
%When $c\in[0,1/2)$ we still get that firms can coexist with the Nash equilibrium in prices given by the one derived for question 2.f, but in the case $c=1/2$ we would have a monopoly of firm 2.

\end{enumerate}

\end{enumerate}


\end{document}
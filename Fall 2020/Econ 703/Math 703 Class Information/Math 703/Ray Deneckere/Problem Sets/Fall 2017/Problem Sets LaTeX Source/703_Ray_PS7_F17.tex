\documentclass[12pt,leqno]{article}
\usepackage{amsmath}
\usepackage{amssymb}
\usepackage{fancyhdr}
\usepackage{float}
\usepackage{mathrsfs}
\usepackage{graphicx}
\usepackage{dsfont}
\textheight 8.5in
\topmargin -.5in

\newcommand{\lgrange}{\mathscr{L}}


\begin{document}

\begin{center}
\Large{Homework {\#}7 Answers}\\
\large{Garrett Anstreicher}
\end{center}

\bigskip
\noindent {\bf Problem 1:}\\
\indent A) This production is Cobb-Douglas, so we know immediately that we will want $x = y = 40,000$. For solving for the $\lambda$ in the Lagrangian equation, we have
$$\lambda = \frac{25\sqrt{y}}{\sqrt{x}} = \frac{25*200}{200} = 25,$$
which will be useful in the next part of the question.

\bigskip
\indent B) Let $V$ be the value function of $Q$ given a particular budget $w$. We know from part A) that a unique solution exists, so the value function is equivalent to the Lagrangian equation evaluated at optimal values for $x, y$ and $\lambda$. Applying the envelope theorem, we can compute how sensitive production is to budget when at budget level 80,000:
$$\frac{\partial V}{\partial w} = \frac{\partial \lgrange}{\partial w}\rvert_{x^*, \lambda^*, \mu^*} = \lambda \rvert_{x^*, \lambda^*, \mu^*} = \lambda^* = 25$$
So, our estimate for the loss of production from reducing the budget by 1,000 is $25*1,000 = 25,000$. 

\bigskip
\indent C) The actual production loss is:
$$50(40,000) - 50(39,500) = 25,000,$$
the same as our estimate. 

\bigskip
\noindent {\bf Problem 2:}\\
\indent A) We can think of this as initially operating in the xyz coordinate plane. Applying the condition that $x^2+y^2 = 1$ restricts our space to a cylinder of radius 1 centered around the origin that spans the entire z-axis. Applying the $x + z = 1$ condition will take a single diagonal slice out of the cylinder as the $x + z = 1$ plane passes through it. This will bound all our values for x, y and z, so a maximum for this problem should exist.

\bigskip
\indent B) 
$$Dg = \begin{bmatrix}
	2x & 2y & 0 \\
	1 & 0 & 1
\end{bmatrix}.$$
This will be linearly dependent when $x = y = 0$, which does not fall in the constraint.

\bigskip
\indent C) Setting up the Lagrangian and taking the five partials and setting them to zero results in the following series of equations:
\begin{eqnarray}
yz - 2x\lambda_1 - \lambda_2 = 0 \\
xy - 2y\lambda_1 = 0 \\
xy - \lambda_2 = 0\\
x^2 + y^2 = 1 \\
x + z = 1
\end{eqnarray}
We first plug $\lambda_2 = xy$ and $\lambda_1 =\frac{xz}{2y}$ into (1) to yield
$$xy = yz - \frac{x^2z}{y}$$
Divide all terms by $y$ and substitute in the identities $y^2 = 1- x^2$ and $z = 1-x$ to yield:
$$x = 1-x - \frac{x^2(1-x)}{1-x^2}$$
$$2x - 1 = \frac{x^2}{1+x}$$
$$2x^2 + x - 1 = -x^2$$
$$3x^2 + x - 1 = 0$$
Applying the quadratic formula yields
$$x = \frac{-1 +/- \sqrt{13}}{6}$$
It turns out that only term that subracts the root thirteen is the maximizer, but even I have a limit as to how much algebra and typesetting I'm willing to do, so I just wolfram alpha'd the the answers for $y$ and $z$: 
$$x, y, z = \left(\frac{-1-\sqrt{13}}{6},\frac{-1}{3}\sqrt{\frac{1}{2}(11-\sqrt{13}}, \frac{7 + \sqrt{13}}{6}\right).$$



\bigskip
\noindent {\bf Problem 3:}\\
\indent The constraint will bind with equality because if we're trying to maximize a sum and can increase the individual elements a little bit, we will certainly want to. So, we can set up the Lagrangian:
$$\lgrange = \sum^T_{t=1} \frac{1}{2^t}\sqrt{x_t} + \lambda\left(1 - \sum^T_{t=1}x_t\right)$$
Take the partial derivative of this expression with respect to an arbitrary $x_t$:
$$\frac{\partial \lgrange}{\partial x_t} = \frac{1}{2^t}\frac{1}{2\sqrt{x_t}} = \lambda \Longrightarrow \sqrt{x_t} = \frac{1}{2^{t+1}\lambda} \Longrightarrow x_t = \frac{1}{4^{t+1}\lambda^2}$$
We can put this expression of $x_t$ into the budget constraint to obtain an expression for $\lambda$:
$$\sum^T_{t=1} \frac{1}{4^{t+1}\lambda^2} = 1 \Longrightarrow \frac{1}{4 \lambda^2} \sum^T_{t=1} \frac{1}{4^t} = 1$$
Invoking the mighty gods of Wolfram Alpha, we expand the sum and obtain:
$$\frac{1}{4\lambda^2}\left(\frac{1 - 4^{-T}}{3} = 1\right) \Longrightarrow \lambda^2 = \frac{1 - 4^{-T}}{12}$$
We finish by substituting in this value for $\lambda^2$ to obtain an expression for choosing an arbitrary $x_t$:
$$x_t = \frac{1}{4^{t+1}\lambda^2} = \frac{12}{4^{t+1}(1 - 4^{-T})}.$$

\bigskip
\noindent {\bf Problem 4:}\\
\indent This utility function will demand that $x_2 = x_3$. As a result, this problem can be verbally expressed as the consumer choosing between spending everything on $x_1$ and buying some $x_1$ and complementing it with equal amounts of $x_2$ and $x_3$. Plugging this problem into a Lagrangian won't go well because Leontief functions aren't differentiable.

\indent It may be possible to use the knowledge that any bundle of consumption that includes $x_2$ must have $x_2=x_3$ to be optimal to try to reframe the problem as:
$$\max_{x_1, x_2} x_1^{1/3} + x_2 \;\; \text{such that} \;\; p_1 x_1 + (p_2 + p_3)x_2 \leq I$$
And try to go from there, but attempting to solve this via Lagrangeans results in some strange equations, perhaps because preferences are convex in one good and linear in another.

\end{document}